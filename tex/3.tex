\section{Konstrukcija grafa}
Apstraktno sintaksno stablo je i samo graf, ali se značajno razlikuje od grafa
kakvog smo opisali u~\ref{sec:uvod}. Primijetimo, još jednom, da
slijed vrhova i bridova u AST-u odgovara slijedu samog koda. Imena i atributi
definirani u njemu su smisleni Pythonovom interpreteru prilikom
dinamičkog izvođenja programa, ali da bismo sami mogli raspoznati imena,
njihove uloge i prodrijeti kroz ljudskom oku ponekad konvoluiranu strukturu AST-a,
potreban je dodatni rad. Zbog toga želimo AST transformirati u 
prikladniji oblik grafa, koji ćemo kasnije prenijeti u grafovsku bazu podataka, a
koja će omogućiti vizualizaciju grafa i izvršavanje upita.

Napomenimo da je izbor konkretne realizacije tog grafa, za koji smo dosad
dali samo inspiraciju, subjektivan. Implicitna shema grafa dana je u~\ref{subsec:implicitna}, 
no u ovom projektu uvijek nastojimo omogućiti jednostavno stvaranje varijanti na glavnu
izvedbu. Doista, za što je moguće temeljitije izvlačenje informacija
iz koda, potrebno je obratiti pažnju na veliki broj specijalnih slučajeva.
Specijalnih slučajeva ne samo da je puno, zbog čega će vjerojatno
uvijek biti novih ideja ili mišljenja, već njihovim uvođenjem
povećavamo kod i kompliciramo njegovu strukturu. Zato se u ovom
trenutku fokusiramo na osnovne funkcionalnosti sa što
elegantnijom strukturom.

Spomenutu transformaciju obavljamo u Pythonu, ali primijetimo
da je neke ideje moguće izvesti i unutar same baze. Kad je to moguće,
dobro je tako napraviti zbog boljih performansi i deklarativnog koda.

\subsection{Obrada jednog modula}
Započnimo s problemom analize jednog modula\footnote{u ovom kontekstu,
izraz \emph{modul} poistovjećujemo s \texttt{.py} datotekom},
 odnosno preslikavanja njegovog koda u graf kakav želimo. U tom problemu imamo
dva glavna cilja:
\begin{itemize}
\item \textit{Obilazak AST-a i izvlačenje informacija o imenima.} Prilikom obilaska AST-a, želimo trenutnom vrhu
dodati atribute (npr.\ tip ako je dan) i povući odgovarajuće bridove s nekim drugim vrhovima u grafu. U
preorder obilasku stabla, obilazimo sintaktičke elemente onim redom kojim su navedeni u kodu. Osim toga,
posebice za povlačenje bridova, potrebno nam je nekakvo praćenje konteksta, budući da u čistom
obilasku u svakom trenutku vidimo samo jedan vrh. U našoj implementaciji koristimo se strukturom
\texttt{SNode} koja osim podataka koje prenosimo u bazu, sprema i neke podatke važne za kontekst.
Pri obilasku koristimo razne \texttt{handler}-e, svaki od kojih je povezan sa specifičnom
vrstom ili vrstama vrha u AST-u. Na taj način enkapsuliramo željene funkcionalnosti unutar
jedne funkcije i dozvoljavamo fleksibilnost po pitanju njihovog izbora. Moguće je pri
obradi jednog vrha dodati njegovu djecu na stog za kasniju obradu, ali ih i ignorirati ili
obratiti odmah ako je to moguče ili prikladno.

\item \textit{Rezolucija imena.} Programski kod sastoji se od raznih imena, a njihovo značenje ovisi ne samo
o dinamičnom pridruživanju već i o nazivnom prostoru u kojemu se nalaze. Ključno je razlikovati dva
imena koja se nalaze u različitim prostorima (npr.\ svaka od dvije funkcije može imati vlastitu lokalnu
varijablu \texttt{x}), ali i prepoznati dva imena kao ista (posebice prilikom uvoza).
\end{itemize}

Ta dva cilja ispunjavamo pomoću tri obilaska:
\begin{enumerate}
\item \textit{Prvi obilazak.} U prvom obilasku prolazimo kroz tablicu simbola.
	Za svaki novi lokalni simbol dodajemo novi \texttt{SNode} i postavljamo
	kontekstualne veze --- roditelja i rječnik s vlastitom djecom kad simbol
	ima vlastiti nazivni prostor.
\item \textit{Drugi obilazak.} U drugom obilasku obilazimo AST i sve atribute dosad viđenih
	simbola dodajemo kao nove simbole.
\item \textit{Treći prolazak.} U trećem obilasku ponovo obilazimo AST. Fokusiramo se na ranije spomenute
	funkcionalnosti važne kod obilaska i uglavnom stvaramo bridove.
\end{enumerate}


\subsection{Proširenje na cijeli paket}
Da bismo analizirali cijeli paket\footnote{u ovom kontekstu, \emph{paket} poistovjećujemo s
mapom (folderom); u prošlosti je mapa
morala imati \texttt{__init__.py} file da se smatra paketom, no u novijim verzijama
($\ge$3.3) nije obavezan}
koji se sastoji od većeg broja modula i potpaketa, potrebno je doraditi dosadašnji algoritam:
\begin{itemize}
\item Najprije je potreban još jedan "nulti" obilazak, ovaj put fileova u repozitoriju, kojim
nalazimo sve module i pakete (\texttt{__init__.py} može definirati uvoz i varijable na razini paketa)
koje moramo analizirati. I ovdje je bitno uspostaviti kontekstualne veze između paketa i modula.

\item Izvršavamo prvi obilazak svakog od modula\footnote{ovdje mislimo i na module i na pakete s
\texttt{__init__.py}-jem}
u proizvoljnom poretku.

\item U drugim obilascima, koje opet izvršavamo u proizvoljnom poretka, osim atributa analiziramo i
uvoz (\emph{import}) dodajući u graf relevantne bridove.
Ovo je bitno da bismo u trećem prolasku mogli uvezena imena ispravno povezati s vrhovima. Naime,
ako modul \texttt{X} uvozi iz \texttt{Y}, a \texttt{Y} iz \texttt{Z}, tada će prvo biti potrebno
uvesti simbole iz \texttt{Z} u \texttt{Y}, budući da \texttt{X} tranzitivno
uvozi i iz \texttt{Z}.

\item Treći obilazak izvršavamo kao ranije, ali u specifičnom poretku --- nikad
ne počinjemo treći obilazak modula prije nego obiđemo one module iz kojih uvozi.
To je moguće pod pretpostavkom da nema cirkularnosti u uvozu, što, iako nije
samo po sebi zabranjeno, može dovesti do nepredividivog ponašanja.

\end{itemize}


