\section{Osvrt}

\subsection{Nedostaci} \label{subsec:nedostaci}
Pythonova dinamičnost i fleksibilnost donose ozbiljne nedostatke ovakvom projektu od
početka. Navedimo tri značajna uzroka:

\begin{enumerate}

	\item \emph{Fleksibilnost u pridruživanju imena.} Kao što smo već spomenuli
u~\ref{subsec:python}, isto ime u toku programa može biti vezano za više
potpuno različitih objekata. Zbog toga, naši rezultati ne moraju u
svim slučajevima biti pouzdani, a mogu ispasti i zbunjujući. Ipak,
takvo recikliranje imena nije dobra praksa i ne očekujemo da ćemo ga
često viđati.

\item \emph{Dinamički generirani kod.} U Pythonu su mogući razni oblici koda
koje možemo smatrati dinamički generiranim. Takav kod je onda teško, ako
uopće moguće, analizirati unutar apstraktnog sintaksnog stabla.
Jedan primjer je Pythonova funkcija \texttt{exec} koja dan joj string
izvodi kao kod. Korištenje te funkcije je rijetkost i najčešće nije
dobra ideja.

Ipak, postoje drugi primjeri koji su puno češći. Primjer su
funkcije \texttt{getattr} i \texttt{setattr} koje upravljaju s
atributima objekta. Pritom se atribut daje kao string, a
ne mora biti \texttt{ast.Name} vrh već može biti rezultat
evaluacije nekog izraza ili iteratora.

Drugi važni primjer su funkcije iz standardne biblioteke
\texttt{importlib} koja omogućava veću kontrolu nad uvozom modula
i paketa. Ponovo, značenje njihovih argumenata može se skrivati
iza evaluacije kompliciranog i dinamički određenog izraza.
Uporaba biblioteke \texttt{importlib} upravo je češća u
sofisticiranijim paketima koji i zahtijevaju takve
dodatne funkcionalnosti. To predstavlja značajan problem unutar
ovog projekta jer ne možemo povezati imena uvezena na taj način
s objektima. Preostaje izbor --- stvarati duplikate za vrhove ili
ignorirati sva imena koja ne možemo rezolvirati. Duplicirani vrhovi
imaju memorijsku cijenu, a sama njihova prisutnost ne garantira
uvijek da će biti kao takvi prepoznati.

\item \emph{Objekti uvezeni iz koda u drugom jeziku.} U Pythonu
je moguće uvesti objekte i imena definirane u drugim 
programskim jezicima, a poznat su i česti slučaj C i
C++. Tada ni ta imena ne možemo rezolvirati bez značajnog
dodatnog napora. Znamenit primjer su Pythonovi \emph{built-in} moduli.
Primjerice, \texttt{ast.py}, koji je zapravo omotač,
uvozi modul \texttt{_ast.py} kreiran iz izvornog koda
u C-u (v.\ \cite{ast_u_c}).

\end{enumerate}

\subsection{Slični projekti} \label{subsec:slicni}

U trenutku pisanja rada, autor nije bio svjestan projekata
koji bi odgovarali ovom u funkcionalnostima, naime
izvlačenju semantičkih pojava u kodu u Pythonu, njihovom modeliranju
grafom i spremanju u grafovsku bazu podataka. Navedimo
dva projekta koji su donekle slični:

\begin{itemize}

\item \texttt{pyclrb.py} (v.\ \cite{docs:pyclbr}). Ovaj modul Pythonove
standardne biblioteke analizira klase i funkcije unutar jednog modula.
Za to se također koristi obilaskom AST-a.

\item \texttt{Pyan} (v.\ \cite{repo:pyan}). Ovaj paket, vjerojatno
najsličniji našem, statički analizira kod u jednoj ili više datoteka i
konstruira graf koji se može vizualizirati pomoću Graphviza. Također se
koristi AST-ovima i tablicama simbola. Osim razlika u implementaciji,
glavna je razlika da je konkretno riječ o statičkom \emph{call graph}-u,
dakle paket se bavi samo funkcijama, metodama, i njihovim međusobnim
pozivanjem.

\end{itemize}

Pored ovih paketa, postoji još obilje drugih koji s vlastitim, specifičnim,
namjenama vrše statičku analizu koda u Pythonu, pritom se
najčešće služeći sličnim alatima.



