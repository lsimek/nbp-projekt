\section{Uvod} \label{sec:uvod}
\subsection{Motivacija}
Suvremeni softverski paketi često u sebi sadrže nepregledno veliku količinu raznih elemenata, od modula, klasa i funkcija do samog broja linija koda. Kao korisnicima, često nam se korisno malo bolje upoznati s arhitekturom paketa s kojim radimo, bilo da je zbog što kvalitetnijeg korištenja njegovih funkcionalnosti ili debuggiranja vlastitog koda.


S tim ciljem, pogodno bi bilo imati način za analizirati strukturu nekog paketa vizualno i potencijalno iz ptičje perspektive.
 Također, htjeli bismo da ta analiza bude fleksibilna --- kao što smo naveli, cjelokupni sadržaj paketa može biti vrlo nepregledan, stoga želimo istovremeno biti u mogućnosti vizualizirati različite module u paketu, ali i podatke o tek nekolicini konkretnih funkcija.


Primijetimo da se spomenuta struktura prirodno preslikava na graf.
Za vrhove grafa uzimamo različite objekte (potpakete, module, klase, funkcije i ostale varijable), dok su bridovi dani njihovim raznim odnosima (klasa je definirana u modulu, funkcija je metoda klase, jedna klasa naslijeđuje od druge itd.) 


Naši zahtjevi --- sakupljanje velike količine podataka o paketu ili više
njih, fleksibilno dohvaćanje raznih dijelova tih podataka i
modeliranje preko grafa navode nas na korištenje grafovske baze podataka. Konkretno, u ovom projektu koristit ćemo Neo4j.
Jedan od razloga za to je da je riječ o trenutno najpoznatijem sustavu
za upravljanje grafovskim bazama podataka. Cypher, njegov jezik za upite,
sličan je poznatom SQL-u u sinktaksi, ali i sam po sebi vrlo kvalitetan i pogodan za upite nad grafovskim bazama.
Dodatno, Neo4j Browser u sebi već ima ugrađeno korisničko sučelje i
mogućnosti vizualizacije rezultata upita.


\subsection{Python i njegove značajke}
Iako različiti programski jezici mogu dijeliti implicitnu shemu takve
grafovske baze, algoritam za transformaciju programskog koda u graf 
nužno će ovisiti o programskom jeziku u kojemu je kod pisan.
Ni zajednička shema nije nužna. Primjerice, već smo nekoliko puta spomenuli klase, no klase nisu dio svih programskih jezika.


U ovom projektu koristit ćemo Python i baviti se paketima pisanima, barem prvenstveno, u Pythonu. Jedan razlog svakako je u popularnosti jezika ---
Python je popularan jezik općenite uporabe i koristi se u
velikom broju otvorenih paketa. Posebno spomenimo cijeli \emph{data science} ekosustav u kojemu je Python zauzeo jednu od vodećih uloga u izradi biblioteka za znanstveno računanje, obradu i vizualizaciju podataka, statistiku i statističko učenje.


Činjenica da je Python dinamički tipiziran predstavlja izazov, ali i povećava smisao ovakvog projekta. U Pythonu ne postoje varijable u tipičnom smislu, već postoje samo \emph{imena} koja možemo pridružiti objektima. To pridruživanje je potpuno fleksibilno. Primjerice, sljedeći kod je sasvim pravilan:
\begin{lstlisting}[language=Python]
class A:
	pass

a = A()
A = 3
\end{lstlisting}

Dakle, dinamički je određeno na koji se objekt odnosti oznaka \texttt{A}.
U ovom projektu zadržavamo se na statičkoj analizi koda --- dinamička analiza bila bi skuplja vremenski i memorijski te bi bila dosta kompleksnija. Dodatno, bila bi ovisna o ispravno postavljenom i funkcionalnom softverskom okruženju. Zauzvrat, naši rezultati ne moraju uvijek biti pouzdani, ali
se situacija kao iznad gotovo nikad neće naći u praksi, među
poznatim i kvalitetno napravljenim paketima.

Treći razlog za Python je njegova opsežna standardna biblioteka koja će nam omogućiti interakciju sa strukturama podataka koje inače koristi samo kompajler. Više o tome ćemo reći u sljedećem poglavlju.

