\section{Baza}

\subsection{Implicitna shema} \label{subsec:implicitna}
U ovom potpoglavlju dajemo konkretnu implicitnu shemu naše grafovske baze,
uz ponovnu napomenu kako je uz modifikacije moguće postići razne varijante na temu,
ovisno o osobnim izborima ili specijaliziranoj namjeni.

Sljedeće su prisutni tipovi (labele) vrhova: \texttt{Package},
\texttt{Module}, \texttt{Class}, \texttt{Function} i \texttt{Name}.
Glavni atribut svih vrhova je \texttt{fullname} koji predstavlja
puno ime objekta iz perspektive korijena paketa. On se razlikuje između
svaka dva različita vrha. Drugi zajednički atributi su \texttt{name},
\texttt{moduleName} i \texttt{packageName} koji su imena
bez prefiksa i olakšavaju neke vrste upita. Daljnji atributi ovise o
tipu --- moduli, klase i funkcije mogu imati \texttt{docstring} (dokumentacijski string),
funkcije imaju atribut \texttt{isAsync} koji govori je li funkcija asinkrona, a
ostala imena mogu imati zabilježen tip (npr.\ \texttt{int} ili \texttt{str}).

Moguće je bilo dodati još tipova (npr.\ koji predstavljaju posebno metode,
atribute ili argumente) no to semantički prikazujemo bridovima na način koji ćemo opisati
u nastavku.

Sada navodimo tipove (labele) bridova:
\begin{itemize}
\item \texttt{WIHIN_SCOPE}. Govori da je prvi vrh definiran u nazivnom prostoru drugog. Implicitno, prvi vrh je ime, klasa ili funkcija,
dok je drugi paket, modul, klasa ili funkcija. Nadalje nećemo uvijek specificirati moguće tipove vrhova.

\item \texttt{ASSIGNED_TO_WITHIN}. Govori da je prvom vrhu pridružena vrijednost unutar opsega drugog.
\item \texttt{REFERENCED_WITHIN}. Govori da je prvi vrh "spomenut" unutar opsega drugog.

\item \texttt{IMPORTED_TO}. Govori da je prvi vrh uvezen u opseg drugog.
\item \texttt{IMPORTS_FROM}. Suprotno, drugi vrh uvezen je u opseg prvog. Ovaj brid nije uvijek
dualan prethodnom jer se odnosi isključivo na module i pakete.

\item \texttt{METHOD}. Označava funkciju metodom klase.
\item \texttt{DECORATES}. Označava da je prvi vrh dekorator drugom.
\item \texttt{ARGUMENT}. Označava da je vrh argument u funkciji.

\item \texttt{ATTRIBUTE}. Označava da je vrh atributut drugog.

\item \texttt{RETURNS}. Govori da se vrh "spominje" u \texttt{return} naredbi funkcije. Ako postoji
logičko grananje ili drugi oblik kontrole toka programa, to se ignorira. Ovaj brid ne govori
što točno funkcija vraća, već samo ističe da postoji veza između vrha i vraćenog.
\item \texttt{ASSIGNED_TO}. Govori da se prvi vrh "spominje" u definiciji drugog. Ponovo,
kontrola toka ili višestruka pridruživanja se ignoriraju. Brid samo ističe da postoji veza,
a ne pokušava otpetljati kontrolu toka, dinamičko stanje ili smisao.

\item \texttt{TYPED_WITH}. Označava da je vrh tipiziran klasom. Ako nije tipiziran klasom
definiranom unutar paketa nego \emph{built-in} tipom, taj podatak se sprema kao atribut vrha.

\end{itemize}

U trenutnoj izvedbi bridovi najčešće nemaju atributa, a iznimka je \texttt{alias} kod uvoza.

\newpage
\subsection{Aplikacija}
