\section{Baza}

\subsection{Implicitna shema} \label{subsec:implicitna}
U ovom potpoglavlju dajemo konkretnu implicitnu shemu naše grafovske baze,
uz ponovnu napomenu kako je uz modifikacije moguće postići razne varijacije na temu,
ovisno o osobnim izborima ili specijaliziranoj namjeni.

Sljedeće su prisutni tipovi (labele) vrhova: \texttt{Package},
\texttt{Module}, \texttt{Class}, \texttt{Function} i \texttt{Name}.
Glavno svojstvo svih vrhova je \texttt{fullname} koje predstavlja
puno ime objekta iz perspektive korijena paketa. Ono se razlikuje između
svaka dva različita vrha. Druga zajednička svojstva su \texttt{name},
\texttt{moduleName} i \texttt{packageName} koji su imena
bez prefiksa i olakšavaju neke vrste upita. Daljnja svojstva ovise o
tipu --- moduli, klase i funkcije mogu imati \texttt{docstring} (dokumentacijski string),
funkcije imaju svojstvo \texttt{isAsync} koje govori je li funkcija asinkrona, a
ostala imena mogu imati zabilježen tip (npr.\ \texttt{int} ili \texttt{str}).

Moguće je bilo dodati još tipova (npr.\ koji predstavljaju posebno metode,
atribute ili argumente) no to semantički prikazujemo bridovima na način koji ćemo opisati
u nastavku.

Sada navodimo tipove bridova:
\begin{itemize}
\item \texttt{WIHIN_SCOPE}. Govori da je prvi vrh definiran u nazivnom prostoru drugog. Implicitno, prvi vrh je ime, klasa ili funkcija,
dok je drugi paket, modul, klasa ili funkcija. Nadalje nećemo uvijek specificirati moguće tipove vrhova.

\item \texttt{ASSIGNED_TO_WITHIN}. Govori da je prvom vrhu pridružena vrijednost unutar opsega drugog.
\item \texttt{REFERENCED_WITHIN}. Govori da je prvi vrh "spomenut" unutar opsega drugog.

\item \texttt{IMPORTED_TO}. Govori da je prvi vrh uvezen u opseg drugog.
\item \texttt{IMPORTS_FROM}. Suprotno, drugi vrh uvezen je u opseg prvog. Ovaj brid nije uvijek
dualan prethodnom jer se odnosi isključivo na module i pakete.

\item \texttt{INHERITS_FROM}. Jedna klasa naslijeđuje drugu.

\item \texttt{METHOD}. Označava funkciju metodom klase.
\item \texttt{DECORATES}. Označava da je prvi vrh dekorator drugom.
\item \texttt{ARGUMENT}. Označava da je vrh argument u funkciji.

\item \texttt{ATTRIBUTE}. Označava da je vrh atributut drugog.

\item \texttt{RETURNS}. Govori da se vrh "spominje" u \texttt{return} naredbi funkcije. Ako postoji
logičko grananje ili drugi oblik kontrole toka programa, to se ignorira. Ovaj brid ne govori
što točno funkcija vraća, već samo ističe da postoji veza između vrha i vraćenog.
\item \texttt{ASSIGNED_TO}. Govori da se prvi vrh "spominje" u definiciji drugog. Ponovo,
kontrola toka ili višestruka pridruživanja se ignoriraju. Brid samo ističe da postoji veza,
a ne pokušava otpetljati kontrolu toka, dinamičko stanje ili smisao.

\item \texttt{TYPED_WITH}. Označava da je vrh tipiziran klasom. Ako nije tipiziran klasom
definiranom unutar paketa nego \emph{built-in} tipom, taj podatak se sprema kao svojstvo vrha.

\end{itemize}

U trenutnoj izvedbi bridovi najčešće nemaju svojstva, a iznimka je \texttt{alias} kod uvoza.

\newpage
\subsection{Aplikacija}
Funkcionalnosti paketa dostupne su preko CLI sučelja. Osnovnoj naredbi \texttt{pygdb}
možemo dodati argumente kojima određujemo URI servera baze, podatke za autentifikaciju
i ime baze. Ukoliko nisu dani, koriste se vrijednosti zadane unutar koda. 

Nakon toga, dostupne su tri podnaredbe (\textsl{subcommands}). Među njima je ključna
\texttt{add} kojom analiziramo paket i dobiveni graf šaljemo u bazu. Detaljnije
o tom problemu raspravljamo u~\ref{subsec:performanse}. Paket je
dan kao URI koji može biti lokacija na lokalnom računalu ili druga mapa 
dohvatljiva naredbom \texttt{git\- clone}. Drugi argument je stupanj \emph{logging}-a.

Nakon što se paket prenese u bazu, upite je moguće vršiti unutar Neo4j Browser sučelja,
ali i pomoću podnaredbe
\texttt{query}. Rezultati upita vizualiziraju se pomoću Graphviz-a. Ta opcija
je potencijalno vrlo spora i nesigurna, dakle treba ju koristiti s oprezom
te eventualno ne izložiti korisnicima s nedovoljnim ovlastima ili za 
iste dodati funkcionalnosti koje bi onemogućile Cypher injection.

Treća podnaredba \texttt{clear} (alternativno \texttt{create}, \texttt{reset} ili \texttt{start})
dozvoljava da se baza resetira, u smislu da se izbriše i stvori nova. Nova baza će se također stvoriti
ako ne postoji, a i dodat će se odabrana ograničenja i indeksi. Dodajemo ograničenje jedinstvenosti svojstva
\texttt{fullname}\footnote{vrhovi se upravo drže u rječniku u kojemu su \texttt{fullname} ključevi,
ali ovo ograničenje je svejedno korisno da spriječi korisnika da zabunom
unese isti paket dvaput; time se na \texttt{fullname} dodaje i indeks}, ograničenje
postojanja svojstva \texttt{name} kao i \emph{fulltext} indeks na istom
budući da očekujemo da će upravo to svojstvo često sudjelovati u upitima.

Za kraj potpoglavlja dajemo jedan primjer potpune naredbe: \\ \\
\texttt{
	python3 pygdb -{}-server bolt:\slash\slash localhost:7689\slash\ \textbackslash \\
	-{}-auth neo4j password -{}-database pygdb \textbackslash \\
	add -{}-uri git@github.com:numpy\slash numpy -{}-name numpy}

\subsection{Prijenos podataka i performanse} \label{subsec:performanse}
Najjednostavniji način za prijenos podataka iz internih
objekata u bazu je stvaranjem zasebne transakcije za svaki vrh
ili brid. Ta metoda ima dodatnu prednost jednostavnog i 
fleksibilnog koda, u smislu da istu funkciju možemo koristiti
za sve tipove vrhova ili bridova i za bilo kakvu konfiguraciju svojstava.
Drugim riječima, u ovoj metodi možemo unositi promjene
u postojeće tipove i svojstva bez da moramo imalo mijenjati kod
za prijenos u bazu. Velika mana ove metode je da je vrlo spora i to
naročito u stvaranju bridova --- njih je više i njihovo stvaranje
zahtijeva dva \texttt{MATCH}-a.

Alternativa je uporaba \emph{batch}-upita kako je preporučeno
i pokazano u~\cite{neo4j:performance}. Ova metoda doista znatno
skraćuje potrebno vrijeme i čini ga podnošljivim i za veće pakete.
Mana je u tome da, kao što je objašnjeno u~\cite{neo4j:params},
parametri ne mogu odgovarati labelama vrha ili tipu brida, nazivu
(ključu) svojstva niti su podržane parametarske mape kao u
prošlom slučaju. Zbog toga je tipove i atribute potrebno
hardkodirati, što komplicira stvaranje izmjena
u implicitnoj shemi.

Druga mogućnost za ostvarivanje boljih performansi je asinkroni driver.
Naravno, veću brzinu ne očekujemo u slučaju mnoštva jednostavnih
transakcija, već ga koristimo samo u \emph{batch} varijanti. Taj driver
je po navodu službene dokumentacije eksperimentalan, stoga postoji
i sinkrona verzija aplikacije u \texttt{sync_\-main.py}.

Razliku u performansama možemo prikazati na primjeru paketa
PyMC (kasnije u~\ref{subsec:pokazna}) s otprilike 66 tisuća bridova.
Dok kraj u prvoj varijanti nije bio ni na vidiku,
u drugoj je cjelokupni prijenos trajao 20 minuta.
Uvođenjem asikronog drivera, to se vrijeme gotovo pa prepolovilo.

